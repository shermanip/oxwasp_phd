\chapter{Expectation of $\widehat{\sigma}^2_0$}
\label{chapter:appendix_expectationNullStdEstimator}

It can be shown that the null variance estimator $\widehat{\sigma}_0^2$ is approximately unbiased when estimating on $n$ $\normal(\mu_0, \sigma_0^2)$ random variables. Suppose the random variables are $Z_1, Z_2, \cdots, Z_n$ and are i.i.d.~with probability density function $f(z)$. Recall that
\begin{equation}
  \widehat{\sigma}_0 = \left[
    \left.
      -\dfrac{\partial^2}{\partial z^2}\ln\widehat{p}(z)
    \right|_{z=\widehat{\mu}_0}
  \right]^{-1/2}
\end{equation}
where
\begin{multline}
  \dfrac{
    \partial^2
  }
  {
    \partial z^2
  }
  \ln\widehat{p}_Z(z)
  =
  \left[
    h\sum_{i=1}^n
    \phi\left(
      \dfrac{
        Z_i-z
      }
      {
        h
      }
    \right)
  \right]^{-2}
  \times
  \left\{
    \left[
      \sum_{i=1}^n
      \phi\left(
        \dfrac{
          Z_i-z
        }
        {
          h
        }
      \right)
    \right]
  \right.
  \\
  \left.
    \times
    \left[
      \sum_{i=1}^n
      \phi\left(
        \dfrac{
          Z_i-z
        }
        {
          h
        }
      \right)
      \left(
        \left(
          \dfrac{
            Z_i-z
          }
          {
            h
          }
        \right)^2
        -1
      \right)
    \right]
    -
    \left[
      \sum_{i=1}^n
      \phi\left(
        \dfrac{
          Z_i-z
        }
        {
          h
        }
      \right)
    \right.
  \right.
  \\
  \left.
    \left.
      \left(
        \dfrac{
          Z_i-z
        }
        {
          h
        }
      \right)
    \right]^2
  \right\}
  \ .
\end{multline}
Apply an approximation such that
\begin{multline}
  \expectation\left[-\widehat{\sigma}_0^{-2}\right]
  \approx
  \left[
    h\sum_{i=1}^n
    \expectation\left[\phi\left(
      \dfrac{
        Z_i-z
      }
      {
        h
      }
    \right)\right]
  \right]^{-2}
  \times
  \left\{
    \left[
      \sum_{i=1}^n
      \expectation\left[
      \phi\left(
        \dfrac{
          Z_i-z
        }
        {
          h
        }
      \right)\right]
    \right]
  \right.
  \\
  \left.
    \times
    \left[
      \sum_{i=1}^n
      \expectation\left[
      \phi\left(
        \dfrac{
          Z_i-z
        }
        {
          h
        }
      \right)
      \left(
        \left(
          \dfrac{
            Z_i-z
          }
          {
            h
          }
        \right)^2
        -1
      \right)
    \right]\right]
    -
    \left[
      \sum_{i=1}^n
      \expectation\left[\phi\left(
        \dfrac{
          Z_i-z
        }
        {
          h
        }
      \right)
    \right.\right.
  \right.
  \\
  \left.
    \left.\left.
      \left(
        \dfrac{
          Z_i-z
        }
        {
          h
        }
      \right)\right]
    \right]^2
  \right\}
  \ .
  \label{eq:appendix_expectationNullStd}
\end{multline}

The following function of $Z$ was studied
\begin{equation}
g_t(Z) = \phi\left(
  \dfrac{Z - z_0}{h}
\right)
\left(
  \dfrac{Z - z_0}{h}
\right)^t
\ .
\end{equation}
The expectation is
\begin{equation}
\expectation\left[g_t(Z)\right] = 
\int_{-\infty}^{\infty}
\phi\left(
  \dfrac{z - z_0}{h}
\right)
\left(
  \dfrac{z - z_0}{h}
\right)^t
f(z) \diff z \ .
\end{equation}
By substituting $u=(z-z_0)/h$ then
\begin{equation}
\expectation\left[g_t(Z)\right] = 
\int_{-\infty}^{\infty}
h\phi(u)u^tf(z_0+uh)\diff u
\end{equation}
which a Taylor series can be used to expand $f(z_0+uh)$
\begin{equation}
\expectation\left[g_t(Z)\right] = 
\int_{-\infty}^{\infty}
h\phi(u)u^t
\sum_{r=0}^\infty \dfrac{
  f^{(r)}(z_0)
  }
  {
  r!
  }
(uh)^r
\diff u
\end{equation}
to get
\begin{equation}
\expectation\left[g_t(Z)\right] = 
\sum_{i=0}^\infty
\dfrac{
  f^{(r)}(z_0)h^{r+1}
}
{
  r!
}
M^{(t+r)}(0)
\end{equation}
where $M(\theta)=\int_{-\infty}^{\infty}\euler^{\theta u}\phi(u)\diff u$ is the moment generating function of a standard Normal random variable. It is left as an exercise to show that
\begin{align}
M(0) &= 1
\\
M^{(1)}(0) & = 0
\\
M^{(2)}(0) & = 1
\\
M^{(3)}(0) & = 0
\\
M^{(4)}(0) & = 3
\\
M^{(5)}(0) & = 0
\\
M^{(6)}(0) & = 15 \ .
\end{align}
Then for $t=0,1,2$
\begin{align}
\expectation\left[g_0(Z)\right]
&=
f(z_0)h
+\dfrac{f^{(2)}(z_0)h^3}{2}
+\dfrac{3f^{(4)}(z_0)h^5}{4!}
+O(h^7)
\label{eq:appendix_expectation1}\\
\expectation\left[g_1(Z)\right]
&=
f^{(1)}(z_0)h^2
+\dfrac{3f^{(3)}(z_0)h^4}{3!}
+O(h^6)
\label{eq:appendix_expectation2}\\
\expectation\left[g_2(Z)\right]
&=
f(z_0)h
+\dfrac{3f^{(2)}(z_0)h^3}{2}
+\dfrac{15f^{(4)}(z_0)h^5}{4!}
+O(h^7)
\label{eq:appendix_expectation3}\ .
\end{align}
Let $f(z)=\dfrac{1}{\sigma_0}\phi\left(\dfrac{z-\mu_0}{\sigma_0}\right)$ then
\begin{align}
f^{(1)}(z) &= \dfrac{f(z)}{\sigma_0^2}\left[-(z-\mu_0)\right]
\\
f^{(2)}(z)&=\dfrac{f(z)}{\sigma_0^2}\left[
  \dfrac{(z-\mu_0)^2}{\sigma_0^2}-1
\right]
\\
f^{(3)}(z) &= \dfrac{f(z)}{\sigma_0^4}
\left[
3(z-\mu_0)-\dfrac{(z-\mu_0)^3}{\sigma_0^2}
\right]
\\
f^{(4)}(z) &= \dfrac{f(z)}{\sigma_0^4}
\left[
  3 - \dfrac{6(z-\mu_0)^2}{\sigma_0^2} + \dfrac{(z-\mu_0)^4}{\sigma_0^4}
\right] \ .
\end{align}
Then substituting these into Equations \ref{eq:appendix_expectation1}, \ref{eq:appendix_expectation2} and \ref{eq:appendix_expectation3} obtains
\begin{multline}
\expectation\left[g_0(Z)\right]
=
f(z_0)h
+
\dfrac{h^3}{2} \dfrac{f(z_0)}{\sigma_0^2}\left[
  \dfrac{(z_0-\mu_0)^2}{\sigma_0^2}-1
\right]
\\ +
\dfrac{3h^5}{4!} \dfrac{f(z_0)}{\sigma_0^4}
\left[
  3 - \dfrac{6(z_0-\mu_0)^2}{\sigma_0^2} + \dfrac{(z_0-\mu_0)^4}{\sigma_0^4}
\right]
+O(h^7)
\label{eq:appendix_expectationExpand1}
\end{multline}
\begin{multline}
\expectation\left[g_1(Z)\right]
=
-h^2\dfrac{f(z_0)}{\sigma_0^2}(z_0-\mu_0)
+
\dfrac{h^4}{2} \dfrac{f(z_0)}{\sigma_0^4}
\left[
3(z_0-\mu_0)-\dfrac{(z_0-\mu_0)^3}{\sigma_0^2}
\right]
\\
+O(h^6)
\label{eq:appendix_expectationExpand2}
\end{multline}
\begin{multline}
\expectation\left[g_2(Z)\right]
=
f(z_0)h
+
\dfrac{3h^3}{2} \dfrac{f(z_0)}{\sigma_0^2}\left[
  \dfrac{(z_0-\mu_0)^2}{\sigma_0^2}-1
\right]
\\ +
\dfrac{15h^5}{4!} \dfrac{f(z_0)}{\sigma_0^4}
\left[
  3 - \dfrac{6(z_0-\mu_0)^2}{\sigma_0^2} + \dfrac{(z_0-\mu_0)^4}{\sigma_0^4}
\right]
+O(h^7) \ .
\label{eq:appendix_expectationExpand3}
\end{multline}

The definition for $\expectation\left[g_t(Z)\right]$ can be used to simplify Equation \ref{eq:appendix_expectationNullStd} to
\begin{equation}
  \expectation\left[\widehat{\sigma}_0^{2}\right]
  =
  \dfrac{
    -\left(h\expectation\left[g_0(Z)\right]\right)^2
  }
  {
    \expectation\left[g_0(Z)\right]\left[
      \expectation\left[g_2(Z)\right] - \expectation\left[g_0(Z)\right]
    \right]
    - \left(\expectation\left[g_1(Z)\right]\right)^2
  }
  \ .
\end{equation}
Substituting in Equations \ref{eq:appendix_expectationExpand1}, \ref{eq:appendix_expectationExpand2} and \ref{eq:appendix_expectationExpand3}  and setting $z_0 = \widehat{\mu}_0$ obtains
\begin{equation}
\expectation\left[\widehat{\sigma}_0^{2}\right]
=
\dfrac{
  \sigma_0^2+h^2\left[\dfrac{(\widehat{\mu}_0-\mu_0)^2}{\sigma_0^2}-1\right]
  +O(h^4)
}
{
  1 + h^2\left[\dfrac{(\widehat{\mu}_0-\mu_0)^2}{\sigma_0^4}-\dfrac{2}{\sigma_0^2}\right]
  +O(h^4)
}
\ .
\end{equation}
This implies that $\widehat{\sigma}_0^{2}$ is an unbiased estimator of $\sigma_0^{2}$ up to the first order approximation. Higher order terms consist of at least even polynomials of $(\widehat{\mu}_0-\mu_0)$ and $h$. As a result, any bias in $\widehat{\mu}_0$ would contribute to the bias of $\widehat{\sigma}_0^{2}$. It is not clear if $\widehat{\sigma}_0^{2}$ is a consistent estimator because $h$ depends on $n$ and the result for $\variance\left[\widehat{\sigma}_0^{2}\right]$ is difficult to obtain.