\chapter{Compound Poisson}

\section{Moment Generating Function}
\label{chapter:appendix_compoundPoissonMgf}

Let $Y\sim\poisson(\lambda)$, $U_1,U_2,U_3,\cdots$ be i.i.d.~random variables and $X|Y=\sum_{i=1}^YU_i$. Let the m.g.f.~of $X$ be $M_X(\theta)=\expectation\left[\euler^{\theta X}\right]$. This can be computed using the result for conditional expectations $M_X(\theta)=\expectation\expectation\left[\euler^{X\theta}|Y\right]$. This results in 
\begin{align}
  M_X(\theta)&=\expectation\expectation\left[\exp\left(\theta U_1 + \theta U_2 + \dotdotdot + \theta U_Y\right)|Y\right]
  \nonumber \\
  &=\expectation\expectation\left[\euler^{\theta U_1}\cdot\euler^{\theta U_2}\cdot\dotdotdot\cdot\euler^{\theta U_Y}|Y\right]
  \ .
\end{align}
But because $U_i$ for $i=1,2,3,\dotdotdot$ are i.i.d., then each $U_i$ has a m.g.f.~$M_U(\theta)=\expectation\left[\euler^{U\theta}\right]$ and
\begin{align}
  M_X(\theta)&=\expectation\left(
    \expectation\left[\euler^{\theta U_1}|Y\right]\cdot
    \expectation\left[\euler^{\theta U_2}|Y\right]\cdot
    \dotdotdot \cdot
    \expectation\left[\euler^{\theta U_Y}|Y\right]
  \right)
  \nonumber\\
  &=\expectation\left[\left(M_U(\theta)\right)^Y\right]
  \nonumber\\
  &=\expectation\left[\euler^{Y\ln(M_U(\theta))}\right]
  \nonumber\\
  & = M_Y\left(\ln(M_U(\theta)\right)
\end{align}
where $M_Y(\theta)=\expectation\left[\euler^{Y\theta}\right]$ is the m.g.f.~of $Y$. It can be shown that the m.g.f.~of $Y$ is
$
  M_Y(\theta)=
  \exp
  \left[
    \lambda
    \left(
      \euler^\theta-1
    \right)
  \right]
$
hence
\begin{equation}
  M_X(\theta)=
  \exp\left[
    \lambda
    \left(
      M_U(\theta)-1
    \right)
  \right]
  \ .
\end{equation}

Moments of $X$ can be obtained from the m.g.f.~by differentiating it and setting $\theta$ to zero
\begin{align}
  M_X'(\theta)&=\exp\left[\lambda\left(M_U(\theta)-1\right)\right]\cdot\lambda M_U'(\theta)
  \nonumber\\
  &=M_X(\theta)\lambda M_U'(\theta)
  \\
  M_X'(0)&=\lambda\expectation\left[U\right]
\end{align}
then
\begin{equation}
  \expectation\left[X\right]=\lambda\expectation\left[U\right]
  \ .
\end{equation}
Conducting the same procedure
\begin{align}
  M_X''(\theta)&=M_X'(\theta)\lambda M_U'(\theta)+M_X(\theta)\lambda M_U''(\theta)
  \nonumber\\
  &=\lambda M_X(\theta)\left[
    \lambda\left(M_U'(\theta)\right)^2+M_U''(\theta)
  \right]
  \\
  M_X''(0)&=\lambda\left[\lambda\left(\expectation[U]\right)^2+\expectation\left[U^2\right]\right]
\end{align}
the variance can be obtained
\begin{align}
  \variance\left[X\right]&=M_X''(0)-\left[M_X'(0)\right]^2
  \nonumber \\
  &=\lambda\left[\lambda\left(\expectation[U]\right)^2+\expectation\left[U^2\right]\right]-\left[\lambda\expectation\left[U\right]\right]^2
  \nonumber \\
  &=\lambda\expectation\left[U^2\right] \ .
\end{align}
Differentiating the m.g.f.~one more time
\begin{align}
  M_X'''(\theta) &=
  \begin{multlined}[t]
    \lambda M_X'(\theta)\left[
      \lambda\left(M_U'(\theta)\right)^2+M_U''(\theta)
    \right]
    \\+
    \lambda M_X(\theta)\left[
      2\lambda M_U'(\theta)M_U''(\theta)+M_U'''(\theta)
    \right]
  \end{multlined}
  \nonumber \\
  &=
  \begin{multlined}[t]
    \lambda^2 M_X(\theta)M_U'(\theta)\left[
      \lambda\left(M_U'(\theta)\right)^2+M_U''(\theta)
    \right]
    \\+
    \lambda M_X(\theta)\left[
      2\lambda M_U'(\theta)M_U''(\theta)+M_U'''(\theta)
    \right]
  \end{multlined}
  \nonumber \\
  &=
  \lambda M_X(\theta)\left[
    \lambda^2 \left(M_U'(\theta)\right)^3 + 3\lambda M_U'(\theta)M_U''(\theta) + M_U'''(\theta)
  \right]
  \\
  M_X'''(0) &=
  \lambda \left[
    \lambda^2 \left(\expectation[U]\right)^3 + 3\lambda \expectation[U]\expectation\left[U^2\right] + \expectation\left[U^3\right]
  \right]
\end{align}
and the third moment about the mean is
\begin{align}
  \expectation\left[\left(X-\expectation[X]\right)^3\right] & =
  M_X'''(0) - 3M_X''(0)M_X'(0) + 2\left(M_X'(0)\right)^3
  \nonumber\\
  &=
  \begin{multlined}[t]
    \lambda \left[
      \lambda^2 \left(\expectation[U]\right)^3 + 3\lambda \expectation[U]\expectation\left[U^2\right] + \expectation\left[U^3\right]
    \right]
    \\-3\lambda\left[\lambda\left(\expectation[U]\right)^2+\expectation\left[U^2\right]\right]\lambda\expectation[U] + 2\left(\lambda\expectation[U]\right)^3
  \end{multlined}
  \nonumber\\
  &=\lambda\expectation\left[U^3\right] \ .
\end{align}

\section{Tweedie Dispersion Exponential Family}
\label{chapter:appendix_tweedie}
Let $X\sim\CPoisson(\lambda,\alpha,\beta)$ and has p.d.f.
\begin{equation}
  p_X(x) = 
  \begin{cases}
    \delta(x) \euler^{-\lambda} & \text{ for } x=0 \\ 
    \displaystyle\sum_{y=1}^{\infty}\dfrac{\beta^{y\alpha}}{\Gamma(y\alpha)}x^{y\alpha-1}\euler^{-\beta x}\euler^{-\lambda}\frac{\lambda^y}{y!} & \text{ for } x>0
  \end{cases}
  \ .
  \label{eq:appendix_compoundPoisson_pdf}
\end{equation}

\cite{jorgensen1987exponential} uses the following parametrisation: $p=\dfrac{2+\alpha}{1+\alpha}$, $\mu=\dfrac{\lambda\alpha}{\beta}$ and $\phi = \dfrac{\alpha+1}{\beta^{2-p}(\lambda\alpha)^{p-1}}$. The parameters can be rearranged
\begin{equation}
  \lambda=\frac{\mu^{2-p}}{\phi(2-p)}
\end{equation}
\begin{equation}
  \alpha=\frac{2-p}{p-1}
\end{equation}
\begin{equation}
  \beta=\frac{1}{\phi(p-1)\mu^{p-1}} \ .
\end{equation}
Substituting these parameters into Equation \eqref{eq:appendix_compoundPoisson_pdf}, the p.m.f.~at zero can be shown to be
\begin{equation}
  \prob(X=0) = \exp
  \left[
      -\frac{\mu^{2-p}}{\phi(2-p)}
  \right]
\end{equation}
and the p.d.f.~for $x>0$ is
\begin{align*}
  p_X(x) &=
  \begin{multlined}[t]
    \sum_{y=1}^{\infty}
    \left[
      \frac{1}{\phi(p-1)\mu^{p-1}}
    \right]^{y\alpha}
    \frac{1}{\Gamma(y\alpha)}
    x^{y\alpha-1}
    \exp\left[
        -\frac{x}{\phi(p-1)\mu^{p-1}}
    \right]
    \\
    \exp\left[
        -\frac{\mu^{2-p}}{\phi(2-p)}
    \right]
    \left[
      \frac{\mu^{2-p}}{\phi(2-p)}
    \right]^y
    \frac{1}{y!}
  \end{multlined}
  \\
  &= 
  \begin{multlined}[t]
    \exp\left[
      \frac{1}{\phi}\left(x\frac{\mu^{1-p}}{1-p}-\frac{\mu^{2-p}}{2-p}\right)
    \right]
    \frac{1}{x}
    \\
    \sum_{y=1}^{\infty}\frac{x^{y\alpha}\mu^{y[2-p-\alpha(p-1)]}}{\phi^{y(1+\alpha)}(p-1)^{y\alpha}(2-p)^yy!\Gamma(y\alpha)}
    \ .
  \end{multlined}
\end{align*}
It should be noted that $2-p-\alpha(p-1) = 2-p - \frac{2-p}{p-1}(p-1) =0$ so that
\begin{equation}
  p_X(x) = 
  \exp\left[
    \frac{1}{\phi}
    \left(
      x\frac{\mu^{1-p}}{1-p}-\frac{\mu^{2-p}}{2-p}
    \right)
  \right]
  \frac{1}{x}
  \sum_{y=1}^{\infty}W_y(x,p,\phi)
\end{equation}
where
\begin{equation}
  W_y(x,p,\phi)=\frac{x^{y\alpha}}{\phi^{y(1+\alpha)}(p-1)^{y\alpha}(2-p)^yy!\Gamma(y\alpha)}
  \ .
\end{equation}

This is in the form of a distribution in the exponential family
\begin{equation}
  p_X(x)=\dfrac{\exp\left(x\theta\right)g(x)}{Z(\theta)}
\end{equation}
where $Z(\theta)$ is the partition function and $\theta$ is the natural parameter. The partition function has some useful properties. It is a normalisation constant $Z=Z(\theta)=\int_{0}^{\infty}\exp(x\theta)g(x)\diff x$. The derivative of the log partition function is
\begin{align*}
  \dfrac{\partial \ln Z}{\partial \theta} &= \dfrac{1}{Z}\dfrac{\partial Z}{\partial \theta}
  \\
  & = \dfrac{1}{Z}\dfrac{\partial}{\partial \theta} \int_0^\infty \exp(x\theta) g(x) \diff x
  \\
  & = \dfrac{1}{Z}\int_0^\infty x \exp(x\theta) g(x) \diff x
\end{align*}
which results in
\begin{equation}
\expectation[X] = \dfrac{\partial \ln Z}{\partial \theta} \ .
\end{equation}
The second derivative of the log partition function is
\begin{align*}
  \dfrac{\partial^2\ln Z}{\partial \theta^2} &=
  \dfrac{
    Z\frac{\partial^2 Z}{\partial\theta^2}-\left(\frac{\partial Z}{\partial\theta}\right)^2
  }{
    Z^2
  }
  \\
  &=\dfrac{1}{Z}\dfrac{\partial^2 Z}{\partial \theta^2} - \left(\dfrac{1}{Z}\dfrac{\partial Z}{\partial \theta}\right)^2
  \\
  &=\dfrac{1}{Z}\dfrac{\partial^2}{\partial \theta^2}\int_0^\infty \exp(x\theta) g(x) \diff x - \left(\expectation\left[X\right]\right)^2
  \\
  &=\dfrac{1}{Z}\int_0^\infty x^2\exp(x\theta) g(x) \diff x - \left(\expectation\left[X\right]\right)^2
  \\
  &=\expectation\left[X^2\right] - \left(\expectation\left[X\right]\right)^2
\end{align*}
thus
\begin{equation}
\variance\left[X\right] = \dfrac{\partial^2\ln Z}{\partial \theta^2} \ .
\label{eq:appendix_variancePartitionFunction}
\end{equation}

For the compound Poisson-gamma distribution, the natural parameter is
\begin{equation}
  \theta = \theta(\mu) = \dfrac{\mu^{1-p}}{\phi(1-p)}
  \label{eq:appendix_cp_naturalParameter}
\end{equation}
and the partition function is
\begin{equation*}
  Z(\theta) = 
  \exp\left[
    \frac{\mu^{2-p}}{2-p}
  \right]
  \ .
\end{equation*}
Rearranging the natural parameter $\mu=(\theta\phi(1-p))^{1/(1-p)}$, the partition function is
\begin{equation}
  Z = \exp\left[
    \phi^{\frac{1}{1-p}}
    \cdot
    \theta^{\frac{2-p}{1-p}}
    \cdot
    \dfrac{
      (1-p)^{\frac{2-p}{1-p}}
    }{
      2-p
    }
  \right]
  \ .
\end{equation}
The log partition function is
\begin{equation}
  \ln Z = \phi^{\frac{1}{1-p}}
    \cdot
    \theta^{\frac{2-p}{1-p}}
    \cdot
    \dfrac{
      (1-p)^{\frac{2-p}{1-p}}
    }{
      2-p
    }
  \ .
\end{equation}
Taking the first derivative
\begin{align}
  \dfrac{\partial \ln Z}{\partial \theta} &=
  \phi^{\frac{1}{1-p}}
  \cdot
  \dfrac{2-p}{1-p}\cdot\theta^{\frac{2-p}{1-p}-1}
  \cdot
  \dfrac{(1-p)^\frac{2-p}{1-p}}{2-p}
  \nonumber \\
  &= \phi^{\frac{1}{1-p}} \cdot \theta^{\frac{1}{1-p}} \cdot (1-p)^\frac{1}{1-p} \ .
\end{align}
Taking the second derivative
\begin{align}
  \dfrac{\partial^2 \ln Z}{\partial \theta^2} &=
  \phi^{\frac{1}{1-p}}
  \cdot
  \dfrac{1}{1-p}\cdot\theta^{\frac{1}{1-p}-1}
  \cdot
  (1-p)^\frac{1}{1-p}
  \nonumber \\
  &=
  \phi^{\frac{1}{1-p}}
  \cdot
  \theta^{\frac{p}{1-p}}
  \cdot
  (1-p)^\frac{p}{1-p} \ .
\end{align}
Substitute in Equation \eqref{eq:appendix_cp_naturalParameter}
\begin{align*}
  \dfrac{\partial^2 \ln Z}{\partial \theta^2} &=
  \phi^{\frac{1}{1-p}}
  \left[\dfrac{\mu^{1-p}}{\phi(1-p)}\right]^{\frac{p}{1-p}}
  (1-p)^\frac{p}{1-p}
  \nonumber \\
  & = \phi \mu^p
\end{align*}
and using the result from Equation \eqref{eq:appendix_variancePartitionFunction}
\begin{equation}
 \variance[X] = \phi \mu^p
\end{equation}
shows that the compound Poisson-gamma distribution is in the Tweedie dispersion exponential family with $1<p<2$.

\section{Normal Approximation}
\label{chapter:appendix_normalApproximation}
Let $X\sim\CPoisson(\lambda,\alpha,\beta)$ with m.g.f.~$M_X(\theta)=\exp\left[\lambda\left(\left(\frac{\beta}{\beta-\theta}\right)^{\alpha}-1\right)\right]$. The m.g.f.~in this form is not useful because when considering $\lambda\rightarrow\infty$ or $\alpha\rightarrow\infty$ then $M_X(\theta)\rightarrow\infty$. Also for $\beta\rightarrow\infty$, then $M_X(\theta)\rightarrow 1$ which is not useful either.

The compound Poisson-gamma random variable $X$ can be standardised to obtain useful limiting results from the m.g.f. Let
\begin{equation}
  Z = \frac{X-\expectation[X]}{\sqrt{\variance[X]}}
\end{equation}
then in the form of $Z = bX+a$,
\begin{equation}
  b = \frac{\beta}{\sqrt{\alpha(\alpha+1)\lambda}}
\end{equation}
and
\begin{equation}
  a = -\sqrt{\frac{\alpha\lambda}{\alpha+1}}
  \ .
\end{equation}

The m.g.f.~of $Z$ is
\begin{align}
  M_Z(\theta)&=\expectation\left[\euler^{Z\theta}\right]
  \nonumber \\
  &=\expectation\left[\euler^{(bX+a)\theta}\right]
  \nonumber \\
  &=\euler^{a\theta}M_X(b\theta)
  \ .
\end{align}
Substituting in $a$ and $b$
\begin{align}
  M_Z(\theta)&=
  \exp\left(
      -\theta\sqrt{\frac{\alpha\lambda}{\alpha+1}}
    \right)
  \exp\left[
    \lambda
    \left(
      \left(
        \frac{\beta}{\beta-\frac{\beta\theta}{\sqrt{\alpha(\alpha+1)\lambda}}}
      \right)^\alpha
      -1
    \right)
  \right]
  \nonumber\\
  &=
  \exp\left(
      -\theta\sqrt{\frac{\alpha\lambda}{\alpha+1}}
    \right)
  \exp\left[
    \lambda
    \left(
      \left(
        \frac{\sqrt{\alpha(\alpha+1)\lambda}}{\sqrt{\alpha(\alpha+1)\lambda}-\theta}
      \right)^\alpha
      -1
    \right)
  \right]
  \nonumber\\
  &=
  \exp\left(
      -\theta\sqrt{\frac{\alpha\lambda}{\alpha+1}}
    \right)
  \exp\left[
    \lambda
    \left(
      \left(
        1-\frac{\theta}{\sqrt{\alpha(\alpha+1)\lambda}} 
      \right)^{-\alpha}
      -1
    \right)
  \right]
  \ .
\end{align}

Using the binomial expansion 
\begin{multline}
  \left(
    1-\frac{\theta}{\sqrt{\alpha(\alpha+1)\lambda}}
  \right)^{-\alpha}
  =
  1+
  \sum_{r=1}^\infty \frac{\theta^r\prod_{s=1}^r(\alpha+s-1)}{(\alpha(\alpha+1)\lambda)^{r/2}r!}
  \\
  \text{for }\frac{\|\theta\|}{\sqrt{\alpha(\alpha+1)\lambda}}<1 \ .
\end{multline}
obtains
\begin{equation}
  M_Z(\theta)=
  \exp\left(
      -\theta\sqrt{\frac{\alpha\lambda}{\alpha+1}}
    \right)
  \exp\left[
    \lambda
    \sum_{r=1}^\infty \frac{\theta^r\prod_{s=1}^r(\alpha+s-1)}{(\alpha(\alpha+1)\lambda)^{r/2}r!}
  \right]
  \ .
\end{equation}
Writing in full the $r=1,2$ terms
\begin{multline*}
  M_Z(\theta)=
  \exp\left(
      -\theta\sqrt{\frac{\alpha\lambda}{\alpha+1}}
    \right)
  \exp\left[
    \theta\sqrt{\frac{\alpha\lambda}{\alpha+1}}
    +\frac{\theta^2}{2}
  \right.
  \\
  \left.
    +\lambda\sum_{r=3}^\infty \frac{\theta^r\prod_{s=1}^r(\alpha+s-1)}{(\alpha(\alpha+1)\lambda)^{r/2}r!}
  \right]
\end{multline*}
and a term cancels out
\begin{equation}
  M_Z(\theta)=
  \exp\left[
    \frac{\theta^2}{2}
    +\sum_{r=3}^\infty
    \frac
      {\theta^r\prod_{s=1}^r(\alpha+s-1)}
      {(\alpha(\alpha+1))^{r/2}r!}
    \lambda^{1-r/2}
  \right]
  \ .
\end{equation}

For large $\lambda$
\begin{equation}
  \lim_{\lambda\rightarrow\infty}M_Z(\theta)=\exp\left[\frac{\theta^2}{2}\right]
\end{equation}
which is the same as the m.g.f.~of a standard Normal variable, therefore
\begin{equation}
  \lim_{\lambda\rightarrow\infty}Z\sim\normal(0,1) \ .
\end{equation}
This should make sense as for high $\lambda$, the Poisson random variable has a high expectation, increasing the number of gamma random variables in a summation. Increasing the number of terms in a summation will trigger the central limit theorem.

For high $\alpha$
\begin{align}
  \lim_{\alpha\rightarrow\infty}
  \frac{
    \prod_{s=1}^r(\alpha+s-1)
  }
  {
    \alpha(\alpha+1))^{r/2}
  }
  &=
  \frac{
    \prod_{s=1}^r\alpha
  }
  {
    \alpha^{r}
  }
  \nonumber\\
  &= 1
  \ .
\end{align}
As a result
\begin{equation}
  \lim_{\alpha\rightarrow\infty}
  M_Z(\theta)=
  \exp\left[
    \frac{\theta^2}{2}
    +\sum_{r=3}^\infty
    \frac{\theta^r}{r!}
    \lambda^{1-r/2}
  \right]
  \ .
\end{equation}
This shows that taking the limit $\alpha\rightarrow\infty$ is not enough to get a Normal limiting distribution. The limit must be accompanied with the limit $\lambda\rightarrow\infty$ to obtain
\begin{equation}
  \lim_{\lambda\rightarrow\infty}\lim_{\alpha\rightarrow\infty}Z\sim\normal(0,1)
  \ .
\end{equation}

Finally it should be noted that $M_Z(\theta)$ is independent of $\beta$. Thus $\beta$ has no effect on the convergence to a Normal limiting distribution.

The above results justify the use of the approximation
\begin{equation}
  X\sim\normal\left(\frac{\lambda\alpha}{\beta},\frac{\lambda\alpha(\alpha+1)}{\beta^2}\right)
\end{equation}
for large $\lambda$. The limiting case where $\lambda\rightarrow 0$, $\alpha\rightarrow 0$ and $\beta\rightarrow 0$ will not be discussed here.

\section{Saddlepoint Approximation}
\label{chapter:appendix_saddlepoint}
Let $X\sim\CPoisson(\lambda,\alpha,\beta)$ with m.g.f.~$M_X(\theta)=\exp\left[\lambda\left(\left(\frac{\beta}{\beta-\theta}\right)^{\alpha}-1\right)\right]$. Let the cumulant generating function of $X$ be $K_X(\theta)=\ln M_X(\theta)$. The saddlepoint approximation is 
\begin{equation}
  p_X(x)\approx\left(2\pi K_X''(s)\right)^{-1/2}\exp\left[K_X(s)-sx\right]
  \label{eq:appendix_saddlepoint}
\end{equation}
where $s=s(x)$ is the solution to the saddle point equation $K_X'(s)=x$.

The cumulant generating function is
\begin{equation}
  K_X(\theta) = \lambda
  \left[
    \left(\frac{\beta}{\beta-\theta}\right)^\alpha-1
  \right]
  \ .
  \label{eq:appendix_cgf}
\end{equation}
Taking the derivative with respect to $\theta$
\begin{equation}
  K_X'(\theta)=\frac{\lambda\alpha\beta^\alpha}{(\beta-\theta)^{\alpha+1}}
\end{equation}
and this is known as the saddlepoint equation. The quantity $s=s(x)$ is the solution to the equation $K_X'(s)=x$, that is
\begin{equation*}
  \frac{\lambda\alpha\beta^\alpha}{(\beta-s)^{\alpha+1}} = x
\end{equation*}
with solution
\begin{equation}
  s = \beta - \left(\frac{\lambda\alpha\beta^\alpha}{x}\right)^{\frac{1}{\alpha+1}}
  \ .
  \label{eq:appendix_saddlepointSolution}
\end{equation}
The second derivative of the cumulant generating function is
\begin{equation}
  K_X''(\theta)=\frac{\lambda\alpha(\alpha+1)\beta^\alpha}{(\beta-\theta)^{\alpha+2}} \ .
  \label{eq:appenfix_2nddiffcgf}
\end{equation}

Substituting Equations \eqref{eq:appendix_cgf} and \eqref{eq:appenfix_2nddiffcgf} into Equation \eqref{eq:appendix_saddlepoint}
\begin{equation*}
  p_X(x)\approx
  \frac{1}{\sqrt{2\pi}}
  \left[
    \frac{
      (\beta-s)^{\alpha+2}
    }
    {
      \lambda\alpha(\alpha+1)\beta^\alpha
    }
  \right]^{1/2}
  \exp\left[
    \lambda\left(\left(\frac{\beta}{\beta-s}\right)^\alpha-1\right)-sx
  \right]
  \ .
\end{equation*}
Substituting in Equation \eqref{eq:appendix_saddlepointSolution}
\begin{multline*}
  p_X(x)\approx
  \frac{1}{\sqrt{2\pi}}
  \left[
    \frac{\left(\beta-\beta+\left(\frac{\lambda\alpha\beta^\alpha}{x}\right)^{\frac{1}{\alpha+1}}\right)^{\alpha+2}}{\lambda\alpha(\alpha+1)\beta^\alpha}
  \right]^{1/2}
  \\
  \exp\left[
    \lambda
    \left(
      \left(
        \frac{\beta}{\beta-\beta+\left(\frac{\lambda\alpha\beta^\alpha}{x}\right)^{\frac{1}{\alpha+1}}}
      \right)^\alpha
      -1
    \right)
    -x\left(
      \beta-\left(\frac{\lambda\alpha\beta^\alpha}{x}\right)^{\frac{1}{\alpha+1}}
    \right)
  \right]
\end{multline*}
simplifying further
\begin{align*}
  p_X(x)&\approx
  \begin{multlined}[t]
    \frac{1}{\sqrt{2\pi}}
    \left[
      \frac{
        \left(\frac{\lambda\alpha\beta^\alpha}{x}\right)^{\frac{\alpha+2}{\alpha+1}}
      }
      {
        \lambda\alpha(\alpha+1)\beta^\alpha
      }
    \right]^{1/2}
    \\
    \exp\left[
      \lambda
      \left(
        \beta^\alpha
        \left(\frac{x}{\lambda\alpha\beta^\alpha}\right)^{\frac{\alpha}{\alpha+1}}
        -1
      \right)
      -x\beta
      +(\lambda\alpha\beta^\alpha)^{\frac{1}{\alpha+1}}x^{1-\frac{1}{\alpha+1}}
    \right]
  \end{multlined}
  \\
  &\approx
  \begin{multlined}[t]
    \frac{1}{\sqrt{2\pi(\alpha+1)}}x^{-\frac{\alpha+2}{2(\alpha+1)}}
    \left(\lambda\alpha\beta^\alpha\right)^{\left(\frac{\alpha+2}{\alpha+1}-1\right)/2}
    \\
    \exp\left[
      \lambda\left(
        \beta^\alpha
        \left(\frac{x}{\lambda\alpha\beta^\alpha}\right)^{\frac{\alpha}{\alpha+1}}
        -1
      \right)
      -x\beta
      +(\lambda\alpha\beta^\alpha)^{\frac{1}{\alpha+1}}x^{\frac{\alpha}{\alpha+1}}
    \right]
  \end{multlined}
  \\
  &\approx
  \begin{multlined}[t]
    \frac{
      \left(
        \lambda\alpha\beta^\alpha
      \right)^{\frac{1}{2(\alpha+1)}}
    }
    {
      \sqrt{2\pi(\alpha+1)}}x^{-\frac{\alpha+2}{2(\alpha+1)}
    } 
    \\
    \exp\left[
      -\lambda
      -x\beta
      +x^{\frac{\alpha}{\alpha+1}}
      \left(
        \frac{
          \lambda\beta^\alpha
        }
        {
          \left(\lambda\alpha\beta^\alpha\right)^{\frac{\alpha}{\alpha+1}}
        }
        +
        \left(
          \lambda\alpha\beta^\alpha
        \right)^{\frac{1}{\alpha+1}}
      \right)
    \right]
  \end{multlined}
  \\
  &\approx
  \begin{multlined}[t]
    \frac{
      \left(
        \lambda\alpha\beta^\alpha
      \right)^{\frac{1}{2(\alpha+1)}}\euler^{-\lambda}
    }
    {
      \sqrt{2\pi(\alpha+1)}}x^{-\frac{\alpha+2}{2(\alpha+1)}
    }
    \euler^{-x\beta}
    \\
    \exp\left[
      x^{\frac{\alpha}{\alpha+1}}
      \left(
        (\lambda\beta^\alpha)^{1-\frac{\alpha}{\alpha+1}}\alpha^{-\frac{\alpha}{\alpha+1}}
        +
        \left(
          \lambda\alpha\beta^\alpha
        \right)^{\frac{1}{\alpha+1}}
      \right)
    \right]
  \end{multlined}
  \\
  &\approx
  \begin{multlined}[t]
    \frac{
      \left(\lambda\alpha\beta^\alpha\right)^{\frac{1}{2(\alpha+1)}}\euler^{-\lambda}
    }
    {
      \sqrt{2\pi(\alpha+1)}
    }
    x^{-\frac{\alpha+2}{2(\alpha+1)}}\euler^{-x\beta}
    \\
    \exp\left[
      x^{\frac{\alpha}{\alpha+1}}
      \left(
        (\lambda\beta^\alpha)^{\frac{1}{\alpha+1}}
        \alpha^{-\frac{\alpha}{\alpha+1}}
        +
        \left(\lambda\alpha\beta^\alpha\right)^{\frac{1}{\alpha+1}}
      \right)
    \right]
  \end{multlined}
  \\
  &\approx
  \begin{multlined}[t]
    \frac{
      \left(\lambda\alpha\beta^\alpha\right)^{\frac{1}{2(\alpha+1)}}\euler^{-\lambda}
    }
    {
      \sqrt{2\pi(\alpha+1)}
    }
    x^{-\frac{\alpha+2}{2(\alpha+1)}}\euler^{-x\beta}
    \\
    \exp\left[
      x^{\frac{\alpha}{\alpha+1}}(\lambda\beta^\alpha)^{\frac{1}{\alpha+1}}
      \left(
        \alpha^{-\frac{\alpha}{\alpha+1}}+\alpha^{\frac{1}{\alpha+1}}
      \right)
    \right]
    \ .
  \end{multlined}
\end{align*}

The expression $\alpha^{-\frac{\alpha}{\alpha+1}}+\alpha^{\frac{1}{\alpha+1}}$ can be simplified by putting the two terms over a common denominator
\begin{align}
  \alpha^{-\frac{\alpha}{\alpha+1}}+\alpha^{\frac{1}{\alpha+1}}
  & = \alpha^{\frac{1}{\alpha+1}}+\frac{1}{\alpha^{\frac{\alpha}{\alpha+1}}}
  \nonumber\\
  & = \frac{\alpha^{\frac{1}{\alpha+1}}\alpha^{\frac{\alpha}{\alpha+1}}+1}{\alpha^{\frac{\alpha}{\alpha+1}}}
  \nonumber\\
  & = \frac{\alpha+1}{\alpha^{\frac{\alpha}{\alpha+1}}}
\end{align}
so that the saddlepoint approximation is 
\begin{equation}
p_X(x)\approx
  \frac{
    \left(\lambda\alpha\beta^\alpha\right)^{\frac{1}{2(\alpha+1)}}\euler^{-\lambda}
  }
  {
    \sqrt{2\pi(\alpha+1)}
  }
  x^{-\frac{\alpha+2}{2(\alpha+1)}}
  \euler^{-x\beta}
  \exp\left[
    x^{\frac{\alpha}{\alpha+1}}
    \frac{(\lambda\beta^\alpha)^{\frac{1}{\alpha+1}}(\alpha+1)}{\alpha^{\frac{\alpha}{\alpha+1}}}
  \right]
  \ .
\end{equation}

\section{Series Evaluation for Compound Poisson-Gamma}
\label{chapter:appendix_compoundPoissonSeries}

Let $X\sim\CPoisson(\lambda,\alpha,\beta)$ with p.d.f.
\begin{equation*}
  p_X(x) = 
  \begin{cases}
    \delta(x) \exp
      \left[
          -\dfrac{\mu^{2-p}}{\phi(2-p)}
      \right] 
    &\text{ for } x=0
    \\ 
    \displaystyle
    \exp\left[
      \frac{1}{\phi}
      \left(
        x\frac{\mu^{1-p}}{1-p}-\frac{\mu^{2-p}}{2-p}
      \right)
    \right]
    \frac{1}{x}
    \sum_{y=1}^{\infty}W_y(x,p,\phi)
    &\text{ for } x>0
  \end{cases}
\end{equation*}
where $p=\dfrac{2+\alpha}{1+\alpha}$, $\mu=\dfrac{\lambda\alpha}{\beta}$, $\phi = \dfrac{\alpha+1}{\beta^{2-p}(\lambda\alpha)^{p-1}}$ and
\begin{equation}
  W_y = W_y(x,p,\phi)=\frac{x^{y\alpha}}{\phi^{y(1+\alpha)}(p-1)^{y\alpha}(2-p)^yy!\Gamma(y\alpha)}
  \ .
\end{equation}
\cite{dunn2005series} approximated the sum by truncation
\begin{equation}
  \sum_{y=1}^\infty W_y \approx \sum_{y=y_\text{l}}^{y_\text{u}}W_y
\end{equation}
where $y_\text{l}<y_{\text{max}}<y_\text{u}$ and $y_{\text{max}}$ is the value of $y$ which maximises $W_y$. \cite{dunn2005series} treated $W_y$ as a continuous and differentiable function of $y$.

It is easier to differentiate $\ln(W_y)$ where
\begin{multline*}
  \ln(W_y) = y\alpha\ln(x)-y(1+\alpha)\ln(\phi)-y\alpha\ln(p-1)\\-y\ln(2-p)-\ln(y!)-\ln\Gamma(y\alpha)
\end{multline*}
\begin{equation}
  \ln(W_y) =
  y
  \ln\left(
    \frac{x^\alpha}{\phi^{1+\alpha}(p-1)^\alpha(2-p)}
  \right)
  -\ln(y!)-\ln\Gamma(y\alpha)
  \ .
\end{equation}
Using Stirling's approximation $\ln(n!)\approx\ln\Gamma(n)\approx n\ln(n!)-n$
\begin{equation}
  \ln(W_y) \approx
  y\ln\left(
    \frac{x^\alpha}{\phi^{1+\alpha}(p-1)^\alpha(2-p)}
  \right)
  -y\ln(y)+y-y\alpha\ln(y\alpha) + y\alpha
  \ .
\end{equation}
Taking the derivative with respect to $y$
\begin{align*}
  \frac{\partial \ln(W_y)}{\partial y} &\approx
  \begin{multlined}[t]
    \ln\left(
      \frac{x^\alpha}{\phi^{1+\alpha}(p-1)^\alpha(2-p)}
    \right)
    -\ln(y)-1+1
    \\
    -\alpha\ln(y\alpha)-\alpha+\alpha
  \end{multlined}
  \\
  &\approx
  \ln\left(
    \frac{x^\alpha}{\phi^{1+\alpha}(p-1)^\alpha(2-p)}
  \right)
  -\ln(y)
  -\alpha\ln(y\alpha)
  \ .
\end{align*}

Setting the derivative to zero
\begin{equation*}
  0 \approx \ln\left(
    \frac{
      x^\alpha
    }
    {
      \phi^{1+\alpha}(p-1)^\alpha(2-p)y_{\text{max}}^{1+\alpha}\alpha^\alpha
    }
  \right)
\end{equation*}
\begin{equation*}
  1 \approx 
  \frac{x^\alpha}{\phi^{1+\alpha}(p-1)^\alpha(2-p)y_{\text{max}}^{1+\alpha}\alpha^\alpha}
\end{equation*}
\begin{equation*}
  y_{\text{max}}^{1+\alpha} \approx 
  \frac{x^\alpha}{\phi^{1+\alpha}(p-1)^\alpha(2-p)\alpha^\alpha}
\end{equation*}
\begin{equation*}
  y_{\text{max}} \approx 
  \frac{1}{\phi}
  \left(
    \frac{x}{(p-1)\alpha}
  \right)^{\frac{\alpha}{1+\alpha}}
  (2-p)^{\frac{-1}{1+\alpha}}
  \ .
\end{equation*}
This can be simplified using the fact that $\alpha=\frac{2-p}{p-1}$, $\frac{1}{1+\alpha} = p-1$ and $\frac{\alpha}{1+\alpha} = 2-p$ then
\begin{equation*}
  y_{\text{max}} \approx 
  \frac{1}{\phi}
  \left(
    \frac{x}{2-p}
  \right)^{2-p}
  (2-p)^{1-p}
\end{equation*}
and finally
\begin{equation}
  y_{\text{max}} \approx \frac{x^{2-p}}{\phi(2-p)}
  \ .
\end{equation}

To verify that $y_\text{max}$ is a maximum, the second derivative can be investigated
\begin{equation}
  \frac{\partial^2\ln(W_y)}{\partial y^2}
  \approx
  -\frac{1}{y}(\alpha+1)
\end{equation}
to see that
\begin{equation}
  \frac{\partial^2\ln(W_y)}{\partial y^2} < 0 \quad \text{for }y=1,2,3,\dotdotdot
\end{equation}
therefore $y_\text{max}$ is a maximum.

\chapter{Expectation of $\widehat{\sigma}^2_0$}
\label{chapter:appendix_expectationNullStdEstimator}

It can be shown that the null variance estimator $\widehat{\sigma}_0^2$ is approximately unbiased when estimating on $n$ $\normal(\mu_0, \sigma_0^2)$ random variables. Suppose the random variables are $Z_1, Z_2, \cdots, Z_n$ and are i.i.d.~with probability density function $f(z)$. Recall that
\begin{equation}
  \widehat{\sigma}_0 = \left[
    \left.
      -\dfrac{\partial^2}{\partial z^2}\ln\widehat{p}(z)
    \right|_{z=\widehat{\mu}_0}
  \right]^{-1/2}
\end{equation}
where
\begin{multline}
  \dfrac{
    \partial^2
  }
  {
    \partial z^2
  }
  \ln\widehat{p}_Z(z)
  =
  \left[
    h\sum_{i=1}^n
    \phi\left(
      \dfrac{
        Z_i-z
      }
      {
        h
      }
    \right)
  \right]^{-2}
  \times
  \left\{
    \left[
      \sum_{i=1}^n
      \phi\left(
        \dfrac{
          Z_i-z
        }
        {
          h
        }
      \right)
    \right]
  \right.
  \\
  \left.
    \times
    \left[
      \sum_{i=1}^n
      \phi\left(
        \dfrac{
          Z_i-z
        }
        {
          h
        }
      \right)
      \left(
        \left(
          \dfrac{
            Z_i-z
          }
          {
            h
          }
        \right)^2
        -1
      \right)
    \right]
    -
    \left[
      \sum_{i=1}^n
      \phi\left(
        \dfrac{
          Z_i-z
        }
        {
          h
        }
      \right)
    \right.
  \right.
  \\
  \left.
    \left.
      \left(
        \dfrac{
          Z_i-z
        }
        {
          h
        }
      \right)
    \right]^2
  \right\}
  \ .
\end{multline}
Apply an approximation such that
\begin{multline}
  \expectation\left[-\widehat{\sigma}_0^{-2}\right]
  \approx
  \left[
    h\sum_{i=1}^n
    \expectation\left[\phi\left(
      \dfrac{
        Z_i-z
      }
      {
        h
      }
    \right)\right]
  \right]^{-2}
  \times
  \left\{
    \left[
      \sum_{i=1}^n
      \expectation\left[
      \phi\left(
        \dfrac{
          Z_i-z
        }
        {
          h
        }
      \right)\right]
    \right]
  \right.
  \\
  \left.
    \times
    \left[
      \sum_{i=1}^n
      \expectation\left[
      \phi\left(
        \dfrac{
          Z_i-z
        }
        {
          h
        }
      \right)
      \left(
        \left(
          \dfrac{
            Z_i-z
          }
          {
            h
          }
        \right)^2
        -1
      \right)
    \right]\right]
    -
    \left[
      \sum_{i=1}^n
      \expectation\left[\phi\left(
        \dfrac{
          Z_i-z
        }
        {
          h
        }
      \right)
    \right.\right.
  \right.
  \\
  \left.
    \left.\left.
      \left(
        \dfrac{
          Z_i-z
        }
        {
          h
        }
      \right)\right]
    \right]^2
  \right\}
  \ .
  \label{eq:appendix_expectationNullStd}
\end{multline}

The following function of $Z$ was studied
\begin{equation}
g_t(Z) = \phi\left(
  \dfrac{Z - z_0}{h}
\right)
\left(
  \dfrac{Z - z_0}{h}
\right)^t
\ .
\end{equation}
The expectation is
\begin{equation}
\expectation\left[g_t(Z)\right] = 
\int_{-\infty}^{\infty}
\phi\left(
  \dfrac{z - z_0}{h}
\right)
\left(
  \dfrac{z - z_0}{h}
\right)^t
f(z) \diff z \ .
\end{equation}
By substituting $u=(z-z_0)/h$ then
\begin{equation}
\expectation\left[g_t(Z)\right] = 
\int_{-\infty}^{\infty}
h\phi(u)u^tf(z_0+uh)\diff u
\end{equation}
which a Taylor series can be used to expand $f(z_0+uh)$
\begin{equation}
\expectation\left[g_t(Z)\right] = 
\int_{-\infty}^{\infty}
h\phi(u)u^t
\sum_{r=0}^\infty \dfrac{
  f^{(r)}(z_0)
  }
  {
  r!
  }
(uh)^r
\diff u
\end{equation}
to get
\begin{equation}
\expectation\left[g_t(Z)\right] = 
\sum_{i=0}^\infty
\dfrac{
  f^{(r)}(z_0)h^{r+1}
}
{
  r!
}
M^{(t+r)}(0)
\end{equation}
where $M(\theta)=\int_{-\infty}^{\infty}\euler^{\theta u}\phi(u)\diff u$ is the moment generating function of a standard Normal random variable. It is left as an exercise to show that
\begin{align}
M(0) &= 1
\\
M^{(1)}(0) & = 0
\\
M^{(2)}(0) & = 1
\\
M^{(3)}(0) & = 0
\\
M^{(4)}(0) & = 3
\\
M^{(5)}(0) & = 0
\\
M^{(6)}(0) & = 15 \ .
\end{align}
Then for $t=0,1,2$
\begin{align}
\expectation\left[g_0(Z)\right]
&=
f(z_0)h
+\dfrac{f^{(2)}(z_0)h^3}{2}
+\dfrac{3f^{(4)}(z_0)h^5}{4!}
+O(h^7)
\label{eq:appendix_expectation1}\\
\expectation\left[g_1(Z)\right]
&=
f^{(1)}(z_0)h^2
+\dfrac{3f^{(3)}(z_0)h^4}{3!}
+O(h^6)
\label{eq:appendix_expectation2}\\
\expectation\left[g_2(Z)\right]
&=
f(z_0)h
+\dfrac{3f^{(2)}(z_0)h^3}{2}
+\dfrac{15f^{(4)}(z_0)h^5}{4!}
+O(h^7)
\label{eq:appendix_expectation3}\ .
\end{align}
Let $f(z)=\dfrac{1}{\sigma_0}\phi\left(\dfrac{z-\mu_0}{\sigma_0}\right)$ then
\begin{align}
f^{(1)}(z) &= \dfrac{f(z)}{\sigma_0^2}\left[-(z-\mu_0)\right]
\\
f^{(2)}(z)&=\dfrac{f(z)}{\sigma_0^2}\left[
  \dfrac{(z-\mu_0)^2}{\sigma_0^2}-1
\right]
\\
f^{(3)}(z) &= \dfrac{f(z)}{\sigma_0^4}
\left[
3(z-\mu_0)-\dfrac{(z-\mu_0)^3}{\sigma_0^2}
\right]
\\
f^{(4)}(z) &= \dfrac{f(z)}{\sigma_0^4}
\left[
  3 - \dfrac{6(z-\mu_0)^2}{\sigma_0^2} + \dfrac{(z-\mu_0)^4}{\sigma_0^4}
\right] \ .
\end{align}
Then substituting these into Equations \ref{eq:appendix_expectation1}, \ref{eq:appendix_expectation2} and \ref{eq:appendix_expectation3} obtains
\begin{multline}
\expectation\left[g_0(Z)\right]
=
f(z_0)h
+
\dfrac{h^3}{2} \dfrac{f(z_0)}{\sigma_0^2}\left[
  \dfrac{(z_0-\mu_0)^2}{\sigma_0^2}-1
\right]
\\ +
\dfrac{3h^5}{4!} \dfrac{f(z_0)}{\sigma_0^4}
\left[
  3 - \dfrac{6(z_0-\mu_0)^2}{\sigma_0^2} + \dfrac{(z_0-\mu_0)^4}{\sigma_0^4}
\right]
+O(h^7)
\label{eq:appendix_expectationExpand1}
\end{multline}
\begin{multline}
\expectation\left[g_1(Z)\right]
=
-h^2\dfrac{f(z_0)}{\sigma_0^2}(z_0-\mu_0)
+
\dfrac{h^4}{2} \dfrac{f(z_0)}{\sigma_0^4}
\left[
3(z_0-\mu_0)-\dfrac{(z_0-\mu_0)^3}{\sigma_0^2}
\right]
\\
+O(h^6)
\label{eq:appendix_expectationExpand2}
\end{multline}
\begin{multline}
\expectation\left[g_2(Z)\right]
=
f(z_0)h
+
\dfrac{3h^3}{2} \dfrac{f(z_0)}{\sigma_0^2}\left[
  \dfrac{(z_0-\mu_0)^2}{\sigma_0^2}-1
\right]
\\ +
\dfrac{15h^5}{4!} \dfrac{f(z_0)}{\sigma_0^4}
\left[
  3 - \dfrac{6(z_0-\mu_0)^2}{\sigma_0^2} + \dfrac{(z_0-\mu_0)^4}{\sigma_0^4}
\right]
+O(h^7) \ .
\label{eq:appendix_expectationExpand3}
\end{multline}

The definition for $\expectation\left[g_t(Z)\right]$ can be used to simplify Equation \ref{eq:appendix_expectationNullStd} to
\begin{equation}
  \expectation\left[\widehat{\sigma}_0^{2}\right]
  =
  \dfrac{
    -\left(h\expectation\left[g_0(Z)\right]\right)^2
  }
  {
    \expectation\left[g_0(Z)\right]\left[
      \expectation\left[g_2(Z)\right] - \expectation\left[g_0(Z)\right]
    \right]
    - \left(\expectation\left[g_1(Z)\right]\right)^2
  }
  \ .
\end{equation}
Substituting in Equations \ref{eq:appendix_expectationExpand1}, \ref{eq:appendix_expectationExpand2} and \ref{eq:appendix_expectationExpand3}  and setting $z_0 = \widehat{\mu}_0$ obtains
\begin{equation}
\expectation\left[\widehat{\sigma}_0^{2}\right]
=
\dfrac{
  \sigma_0^2+h^2\left[\dfrac{(\widehat{\mu}_0-\mu_0)^2}{\sigma_0^2}-1\right]
  +O(h^4)
}
{
  1 + h^2\left[\dfrac{(\widehat{\mu}_0-\mu_0)^2}{\sigma_0^4}-\dfrac{2}{\sigma_0^2}\right]
  +O(h^4)
}
\ .
\end{equation}
This implies that $\widehat{\sigma}_0^{2}$ is an unbiased estimator of $\sigma_0^{2}$ up to the first order approximation. Higher order terms consist of at least even polynomials of $(\widehat{\mu}_0-\mu_0)$ and $h$. As a result, any bias in $\widehat{\mu}_0$ would contribute to the bias of $\widehat{\sigma}_0^{2}$. It is not clear if $\widehat{\sigma}_0^{2}$ is a consistent estimator because $h$ depends on $n$ and the result for $\variance\left[\widehat{\sigma}_0^{2}\right]$ is difficult to obtain.